\section{IP (Network Layer)}

\subsection{Router/Gateway}
Verbinden verschiedene Netze mit potentiell unterschiedlichen Technologien

Lösen 2 Aufgaben:
\begin{description}
	\item[Routing] Aufbau und Update von Routingtabellen
	\item[Forwarding] Weiterleiten der Pakete anhand Routingtabellen
\end{description}

\subsubsection{Routing-Tabelle}
löst: wie Kann ich welches Netz erreichen?

Bei hierarchischem Routing (Router wissen welche Netze an anderen Router sind
forwarden entsprechend) kommen oft auch aggregierte Routen vor.\\
Wenn so z.B. 130.0.0.0/25 und 130.0.128.0/25 beide via den selben Router
erreichbar sind wird diese Route aggregiert und als 130.0.0.0/24 zusammengefasst.

\begin{center}
	\includegraphics[width=0.9\linewidth]{routing-table}
\end{center}

\subsubsection{Classful Routing}

Ursprünglich wurden IP Adressen in 5 Routing Klassen eingetelt (Classful Routing).
Die ersten vier Adressbits erlauben eine Bestimmung der Klasse.
Wird nicht mehr gemacht weil oftmals Adressraum verschwendet wird.

\begin{center}
	\includegraphics[width=0.9\linewidth, height=15mm]{classful-routing}
\end{center}

\subsubsection{Subnetting}
\begin{itemize}
	\item das Netz in kleinere Subnetze teilen
	\item funktioniert simplifiziert durch beliebiges erweitern der Netzadresse
	\item hintereinanderliegende Netze können zusammengefügt werden
	      \begin{center}
		      \includegraphics[width=0.8\linewidth, height=10mm]{subnetting-aggr}
	      \end{center}
	      % \begin{itemize}
	      %  \item 160.85.100.0/24
	      %  \item 160.85.101.0/24
	      %  \item 160.85.102.0/24
	      %  \item 160.85.103.0/24
	      %  \item zusammengefügt: 160.85.100/22
	      % \end{itemize}
\end{itemize}








\subsection{Adressierung / IPv4}

Adresse eines Host = Netz-Adresse + Interface-Adresse

\subsubsection{Subnetzmaske}

\begin{center}
	\includegraphics[width=0.9\linewidth]{subnetmask}
\end{center}

\subsubsection{Netzadresse}

\begin{itemize}
	\item reserviert, darf NICHT für interfaces verwendet werden
	\item tiefste Adresse im Subnet $\rightarrow$ alle interface bits 0
	\item berechnung durch $Interface Adresse \land Subnetzmaske$
\end{itemize}

\subsubsection{Broadcast-Adresse}

\begin{itemize}
	\item reserviert, darf NICHT für interfaces verwendet werden
	\item höchste Adresse im Subnet $\rightarrow$ alle interface bits 1
	\item berechnung durch $Interface Adresse \lor invertierte \, Subnetzmaske$
\end{itemize}

\subsubsection{Private Adressen}

Die $172.0.0.0/8$ Adressen sind reserviert für Loopback und verlassen den Host nicht.
Sie werden an ein emuliertes Loopback-Gerät geschickt dass direkt returned
(kein Interface nötig).
\begin{center}
	\includegraphics[width=0.9\linewidth, height=10mm]{private-ip}
\end{center}



\subsection{IPv4 Header}

\textcolor{red}{immer minimum 4Byte Blöcke}

\begin{center}
	\includegraphics[width=0.9\linewidth]{ip4-header}
\end{center}

\begin{description}
	\item[Version] 4 oder 6
	\item[IHL] Internet Header Length ( / 4 weil immer 4 Byte Blöcke)
		\begin{itemize}
			\item Header ohne optionen = 20 Bytes $\rightarrow$ IHL = 5
			\item Maximalwert 15 (4 Bits)
		\end{itemize}
	\item[TOS] Type of Service, erlaubt prioriesierung
		\begin{center}
			\includegraphics[width=0.9\linewidth, height=10mm]{ip4-tos}
		\end{center}
	\item[Total Length] inclusive Header \& Nutzdaten
	\item[TTL] Time To Live, In Anzahl Hops. Verhindern Loops, jeder Router
		dekrementiert, bei 0 $\rightarrow$ Paket verwerfen
	\item[Protocol] Protokoll der Nutzdaten
		\begin{enumerate}
			\item[1] ICMP Internet Control Message Protocol
			\item[6] TCP Transport Control Protocol
			\item[17] UDP User Datagramm Protocol
		\end{enumerate}
	\item[Header Checksum] schütz NUR den Header, bei jedem Router neu berechnet(TTL)
	\item[Options/Padding] variabel, heute selten, Padding für 32 bits
	\item[Identification Number] identifikation, bleibt für fragmentierte Pakete gleich
	\item[Flags] 3 Bits, steuert Fragmentierung über einzelne Bits
		(Aufzählung von links aus)
		\begin{enumerate}[start=0]
			\item reserviert, immer 0
			\item DF, 0/1 $\rightarrow$ May / Dont Fragment
			\item MF, 0/1 $\rightarrow$ Last / More Fragments
		\end{enumerate}
	\item[Fragment Offset] 13 Bits, steht für Anzahl 8 Byte Blöcke, bestimmt
		wo im gesamt paket die Daten hingehören
\end{description}


\subsubsection{Fragmentierung}

IP-Paket maximal 65535 Bytes, aber limitiert durch MTU des Netz. \\
Problem: spätere Netze haben eventuell tiefere MTU $\rightarrow$ Fragmentierung.

\begin{itemize}
	\item jedes Fragment erhält seinen eigenen Header
	\item Identification Number bleibt gleich
	\item Total Length jeweils die Länge des Paket
	\item alle Pakete ausser letztes haben MF = 1 in den Flags
    \item alle Pakete haben die gleiche maximale Länge (ausser letzes)
\end{itemize}
\begin{center}
	\includegraphics[width=0.9\linewidth, height=10mm]{ip4-fragment-ex}
\end{center}








