\section{Data Link Layer}

Aufgaben:
\begin{itemize}
	\item fehlerfreie Verbindung direkt verbundener Systeme
	\item Framing, Nutzdaten erkennen und ent-/verpacken
	\item Fluss-Steuerung (Flow Control), langsamer Empfänger kannen schnellen
	      Sender bremsen
	\item Adressierung der Teilnehmer (falls mehrere)
	\item Medium Zugriff (Media Access) wer darf wann senden
\end{itemize}

\subsection{Framing}

\subsubsection{Asynchron}
\begin{itemize}
	\item keine Daten $\rightarrow$ nichts wird übertragen (Ruhezustand)
	\item Start-Bit zu Beginn eines Frames
	\item Aufbau (meistens): Header + Datenblock + Fehlererkennung
\end{itemize}


\subsubsection{Synchron}
\begin{itemize}
	\item Frames werden permanent gesendet, falls keine daten wird ein Flag gesendet
	      (Bit Stuffing stellt sicher dass das Flag nicht als Nutzdaten vorkommt)
	\item Aufbau: wie Asynchron aber links \& rechts durch Start-/End-Flag begrenzt
\end{itemize}


\subsubsection{Bit-/Framefehlerwarscheinlichkeit}
\small{auch \textbf{BER} (Bit Error Ratio), NICHT Bitfehlerrate das wären Bitfehler pro Zeit}

Für gleichverteilte Fehler, ohne doppelte die sich aufheben; Wahrscheinlichkeit, dass ein Frame mit
länge $N \ge 1$ Bitfehler hat

$$FER = P_{Fehler Frame} = 1 - (1 - p_{Fehler Bit})^N $$
$$ \text{für } p_{Fehler Bit} \ll 1 \text{ gilt näherungsweise }
	FER \approx N * p_{Fehler Bit}$$

Andere:
\begin{description}
	\item[FER] Frame Error Ratio, fehlerhaft empfangene Frames
	\item[RER] Residual Error Ratio, unentdeckete fehlerhaft empfange Frames
\end{description}

\subsubsection{Frame-Länge}
Trade-Offs von Frame länge
\begin{itemize}
	\item[+] hohe Netto-Bitrate wegen Gutem Header/Nutzdaten Verhältnis
	\item[-] grössere Framefehlerwarscheinlichkeit
	\item[-] mehr Datenverlust bei einem Fehler
	\item[-] höhere Wahrscheinlichkeit für unentdeckten Fehler
\end{itemize}

\subsection{Fehlererkennung}

\subsubsection{Hamming-Distanz}
die minimale Anzahl Bits in denen sich zwei beliebige Codewörter eines
Codes unterscheiden

\textcolor{orange}{Hamming-Distanz h herlaubt die Erkennung von h-1 Fehlern}

\subsubsection{Parity}
\begin{description}
	\item[Even] Anzahl 1 inkl. Parity-Bit ist gerade
	\item[Odd] Anzahl 1 inkl. Parity-Bit ist ungerade
		% TODO folie 25,26 was sind resultate der Fragen?
	\item[Längs \& Quer] über mehrere Bytes für jede Spalte / Zeile Parity
		sowie eine gesamt parity
\end{description}

\subsubsection{Prüfsumme}

Hamming-Distanz = 2, z.B. CRC


\subsection{Fehlerkorrektur}

\subsubsection{Backward Error Correction}
Rückfrage des Empfängers nach dem letzen Packet bei Fehler
\begin{itemize}
	\item[-] Rückkanal benötigt
	\item[-] Warten auf Quittung
	\item[-] Warten auf Timeout (abhilfe durch negative Bestätigung)
\end{itemize}

\subsubsection{Forward Error Correction (FEC)}
BER reduzieren durch fehlerkorrigierende Codes

Anstatt verwerfen bei Fehler, schätzen der wahrscheinlichsten gesendeten Nachricht.

Regeln für Hamming-Distanz h
\begin{itemize}
	\item Anzahl der korrigierbaren Bitfehler $k$ ist die abgerundete Hälfte
	      der fehlerbehafteten Zwischencodes $k \le (h-1)/2$
	\item Ein Code der genau in der Mitte liegt wird als Fehlerhaft erkennt,
	      kann aber nicht korrigiert werden.
	\item Allgemein: $k + e = (h-1)$ mit $k \le (h-1)/2$
\end{itemize}


\subsection{Medienzugriff}
\begin{description}
	\item[Master-Slave] \,
		\begin{itemize}
			\item[+] keine Konflikte weil master alles koordiniert
			\item[-] single point of failure
		\end{itemize}
	\item[Token] Sendeberechtigung wird in einer fixen Reihenfolge weitergereicht
		\begin{itemize}
			\item[+] deterministisch
			\item[-] Aufwändig
		\end{itemize}
	\item[Zeitsteuerung] Analog zu Taktfahrplan für Züge
		\begin{itemize}
			\item[+] Optimierung möglich
			\item[-] konflikte bei unplanbarem Verkehr
		\end{itemize}
	\item[CSMA] (Carrier Sense Multiple Access) Alle ''master'', senden wenn sie niemand
		sonst hören
		\begin{itemize}
			\item[-] Konflikte möglich falls 2+ gleichzeitig entscheiden zu senden
		\end{itemize}
	\item[CSMA/CD] Collision Detection, abbrechen und später nochmals versuchen
	\item[CSMA/CR] Collision Resolution, Hardware unterstütze Arbitrierung
\end{description}

\subsection{Flowcontrol}

Erlaubt Empfänger den Sender temporär zu stoppen
Verwendung bei Speichermangel/langsamer Verarbeitung oder Überlast im Netzwerk

Kann auch implizit vorkommen, in dem der Sender nach jeder Nachricht auf eine
Quittung wartet bevor er die nächste schickt








